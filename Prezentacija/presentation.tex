% !TEX encoding = UTF-8 Unicode
\documentclass{beamer}

\usepackage{color}
\usepackage{url}
\usepackage[T2A]{fontenc}
\usepackage[utf8]{inputenc}
\usepackage{graphicx}
\usepackage[serbian]{babel}
\usepackage{chemfig}
\usepackage[version=3]{mhchem}
\usepackage{multicol}

\mode<presentation>
{
  \usetheme{Warsaw}       % or try default, Darmstadt, Warsaw, ...
  \usecolortheme{default} % or try albatross, beaver, crane, ...
  \usefonttheme{serif}    % or try default, structurebold, ...
  \setbeamertemplate{navigation symbols}{}
  \setbeamertemplate{caption}[numbered]
} 


\definecolor{mygreen}{rgb}{0,0.6,0}
\definecolor{mygray}{rgb}{0.5,0.5,0.5}
\definecolor{mymauve}{rgb}{0.58,0,0.82}

\usepackage{listings}
\lstset{ 
  backgroundcolor=\color{white},
  basicstyle=\scriptsize\ttfamily,
  breakatwhitespace=false,
  breaklines=true,
  captionpos=b,
  commentstyle=\color{mygreen},
  deletekeywords={...},            
  escapeinside={\%*}{*)},          
  extendedchars=true,
  firstnumber=1,              
  frame=single,	                
  keepspaces=true,
  keywordstyle=\color{blue},     
  language=Python,                
  morekeywords={*,...},
  numbers=left, 
  numbersep=5pt,                  
  numberstyle=\tiny\color{mygray}, 
  rulecolor=\color{black},
  showspaces=false,
  showstringspaces=false,
  showtabs=false,
  stepnumber=1, 
  stringstyle=\color{mymauve},
  tabsize=2,
  title=\lstname
}
\usepackage{pgfpages}
\pgfpagesuselayout{resize to}[%
  physical paper width=8in, physical paper height=6in]


% Here's where the presentation starts, with the info for the title slide
\title{F\# na .NET platformi}
\author{T.Todorov T.Garibovic D.Nedeljkovic M.Vicentijevic}
\date{\today}

\begin{document}

\begin{frame}
  \titlepage
\end{frame}

% These three lines create an automatically generated table of contents.
\begin{frame}{Outline}
  \tableofcontents
\end{frame}

\section{Uvod}

\begin{frame}{Uvod}

\end{frame}

\section{Istorijat}
\begin{frame}{Istorijat}
 
Nastao je 1970 god. \\
Osnova F\#-a je ML({\em Meta-Language}) jezik baziran na programskom jeziku LISP koji je koren strogo tipizirnih programskih jezika. \\
U decembru 2001. godine u razmatranje vraca jezik {\em OCaml} i razvija projekat Caml.NET koji ce se kasnije preimenovati u F\#. \\
Godine 2002. pojavljuje se prva verzija F\# 0.5 koja je bila slabo primecena. {\em Don Syme} 2004. godine nastavlja intenzivno da razvija ovaj jezik, a pocetkom 2005. godine izbacuje prvu potpunu verziju F\#-a. \\
Poslednja aktuelna verzija jezika je F\# 4.1.

\end{frame}

\section{Primena i mogucnosti}
\begin{frame}{Primena i mogucnosti}

F\# ima svedenu sintaksu koja omogucava laku citljivost koda i koristi se za resavanje slozenih matematickih algoritama.  \\
F\# danas ima siroku primenu u obradi baza podataka, finansijskog modelovanja, statistici i bioinformatici. \\
F\# podrzava sledece paradigme: funkcionalnu, imperativnu, objektno-orjentisanu, paralelnu, distribuiranu, asinhronu, meta programiranje, veb programiranje, skript programiranje, analiticko programiranje... \\
Sistemi na kojima je podrzan: Linux, MAC, Windows, Android, iOS... 
\end{frame}

\section{Osobine i specificnosti jezika}
\begin{frame}{Osobine i specificnosti}
\begin{center}
Osobine:						Specificnosti:
\begin{multicols}{2}
\begin{itemize}
  \item Bezbedan
  \item Funkcionalan
  \item Strogo tipiziran
  \item Automatski zakljucuje tipove
  \item Kompatibilan
  \item Povratna vrednost if/else
  \item Opciono - return
  \item Kljucne reci let i mutable
  \item Pattern matching
  \item Novi tip option type
\end{itemize}
\end{multicols}
\end{center}
\end{frame}

\section{Funcionalna paradigma - Pattern matching}  
\begin{frame}[fragile]
\frametitle{Funcionalna paradigma - Pattern matching}

\begin{itemize}
  \item Glavna paradigma programskog jezika F\#
\end{itemize}

\begin{block}{Pattern matching}
Pattern matching je mehanizam koji koristi dekompoziciju i kontrolu toka podataka za poklapanje obrazaca koriscenjem navedene konstrukcije:
\textbf{ match ... with ...}
\begin{lstlisting}
let urlFilter url agent =
 match (url,agent) with
 | "http://www.control.org", 99 -> true
 | "http://www.kaos.org" , _ -> false
 | _, 86 -> true
 | _ -> false
\end{lstlisting} 
\end{block}

\end{frame}

\section{Asinhrono i paralelno programiranje}
\begin{frame}[fragile]
\frametitle{Asinhrono i paralelno programiranje}



\end{frame}

\section{Radni okvir - .NET framework}
\begin{frame}[fragile]
\frametitle{Radni okvir - .NET framework}
Temelj .NET platforme je zajednicka jezicka infrastruktura CLI({\em(Common Language Infrastructure)}.Kodovi se prevode na MSIL({\em Microsoft Intermediate Language}) alemblerski jezik.\\
Implementacija MSIL-a na CLI kompajleru je brza i ima sledece prednosti u odnosu na masinski:
\begin{itemize}
	\item kompatibilnost medju jezicima
	\item mogucnost rada na vise platformi
	\item masinska nezavisnost
\end{itemize}
Mogucnost automatskog prikupljanja smeca je jos jedna prednost.
Jos neki okviri: veb radni okviri({\em Suave, Fable, ASP.NET Core}...) i radni okviri za testiranje veba ({\em Web Testing, Frameworks, Unit Testing Libraries}...)
\end{frame}

\section{Instalacija i pokretanje}
\begin{frame}[fragile]
\frametitle{Instalacija i pokretanje}

\begin{itemize}
\item Alati koji na Windows-u podrzavaju F\# se instaliraju u nekoliko koraka:
	\begin{itemize}
	\item Visual Studio Code
	\item Visual studio
	\item JetBrains Rider
	\end{itemize}
\item Na Linux-u se instalacija vrsi na isti nacin za sledece verzije:
	\begin{itemize}
	\item Ubuntu
	\item Mint
	\item Debian
	\end{itemize}
	\item \begin{lstlisting}
fsharpc primer.fs
\end{lstlisting}
\end{itemize}
\end{frame}

\section{Fizz Buzz}
\begin{frame}[fragile]
\frametitle{Fizz Buzz}

\begin{lstlisting}
let (|Fizz|Buzz|FizzBuzz|Other|) n =
    match (n % 3, n % 5) with
    | 0, 0 -> FizzBuzz
    | 0, _ -> Fizz
    | _, 0 -> Buzz
    | _ -> Other n

let fizzBuzz =
    function
    | Fizz -> "Fizz"
    | Buzz -> "Buzz"
    | FizzBuzz -> "FizzBuzz"
    | Other n -> n.ToString()

seq { 1..100 } |> Seq.map fizzBuzz|> Seq.iter (printfn "%s")
\end{lstlisting}

\end{frame}

\section{Jedinica mere}
\begin{frame}[fragile]
\frametitle{Jedinica mere}

\begin{itemize}
  \item Pad orbitera poslatog na Mars 1999.
  \begin{itemize}
  	\item Uzrokovan cinjenicom da je deo softvera koristio numericke, a deo softvera engleske jedinice
  \end{itemize}
  \item Prevencija gresaka na osnovu konteksta primene
  \begin{itemize}
  	\item Numerickim tipovima se pridruzuju metapodaci
  	\item Kompajler na osnovu metapodataka proverava ispravnost
  \end{itemize}
  \item Jedinstveno svojstvo jezika F\#
  \item Primer definisanja jedinice mere  
\begin{lstlisting}
  [<Measure>] type cm
  [<Measure>] type inch  
\end{lstlisting}
\end{itemize}
\end{frame}

\begin{frame}[fragile]
\frametitle{Jedinica mere}

\begin{lstlisting}
[<Measure>] type rsd
[<Measure>] type eur
[<Measure>] type hour
[<Measure>] type week
[<Measure>] type year

let hoursBilledPerWeek = 40.0<hour/week>
let weeksWorkedPerYear = 35.0<week/year>
let rsdPerHour = 1000.0<rsd/hour>
let exchangeRate = 0.008547<eur/rsd>

let eurPerYear = rsdPerHour * hoursBilledPerWeek * weeksWorkedPerYear * exchangeRate
let bonus = 500.0<eur/year>

printfn "%f" (eurPerYear + bonus)
\end{lstlisting}
\cite{progFs}
\end{frame}

\section{Literatura}

\begin{frame}{Literatura}

\bibliography{primer1.bib}

\end{frame}

\end{document}